\chapter{Methods}

\section{Toolchain}

The whole project can be summarized in four single steps or tasks:
\begin{enumerate}
	\item Data: Collect a dataset, for inference, for training and for testing
	\item Model: Get to know at least one model, for inference and for retraining
	\item Tidy: For a given input picture, return an image that displays a repertoire of all found objects in the input picture
	\item App: Turn the former three tasks into an application that can be reached from a web browser.
\end{enumerate}

In order to develop a minimal viable product, all these four tasks have to be solved first. The way of proceeding was to develop a minimal viable product first and then to start with refining the model and adjusting the app. This means going through the full circle first, before adjustments and proceeding into retraining are made.

\section{Project management}

In the beginning of the project, a project management plan had to be developed. The project management plan contains a timeline with all (then known) tasks and a list of all milestones. The project management plan can be seen via this link: \url{https://trello.com/b/srqnMstX/object-detection-in-fine-art-photography}. A picture, showing the whole timeline is shown here:

 \begin{figure}
	\center{\includegraphics[width=\textwidth]
	{img/project-management-plan.png}}
	\caption{\label{fig:project-management-plan} Timeline view of the project management plan}
\end{figure}

To better structure the project, a number of milestones have been chosen:

\begin{enumerate}
	\item Test pretrained models on Google Colab with new data
	\item Build web application prototype
	\item Finetune model with fine art photography data
	\item Test finetuned model on new data
	\item Finish web application with new model
\end{enumerate}

\section{Testing}

In favour of having more time available for the development cycle and retraining of the model, no testing strategy has been selected.